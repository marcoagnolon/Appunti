%----------------------------------------
%Materiale rilasciato sotto licenza Creative Commons BY-NC-SA 4.0

%È consentita la distribuzione e la modifica sotto questa licenza, per scopi non commerciali

%http://creativecommons.org/licenses/by-nc-sa/4.0/deed.it

%Tommaso Tabarelli
%Ottobre 2018
%-----------------------------------------
\documentclass[12pt, english, a4paper]{book}
\usepackage{mathptmx}
\usepackage[T1]{fontenc}
\usepackage[utf8x]{inputenc}
\usepackage[english]{babel}
\usepackage{graphicx}
\usepackage{float}
\usepackage{amsmath}
\usepackage{mathtools}
\usepackage{amsfonts}
\usepackage{bbm}
\usepackage{enumerate}

\usepackage[margin=2cm]{geometry}


\usepackage[pdftitle={Models of Theorethical Physics},pdfauthor={Tommaso Tabarelli}]{hyperref}
\hypersetup{
    colorlinks=true,
    linkcolor=blue,
    urlcolor=magenta,
    linktoc=all
}




\newcommand{\bra}[1]{\left\langle #1 \right |}
\newcommand{\ket}[1]{\left| #1 \right \rangle}
\newcommand{\braket}[2]{\left\langle #1 | #2 \right\rangle}
\newcommand{\vmed}[1]{\left \langle #1 \right \rangle}
\newcommand{\tr}{\operatorname{Tr}}

\newcommand{\tab}



\begin{document}


\begin{titlepage}

\begin{center}
\LARGE{Università degli Studi di Padova}\\
\line(1,0){450}\\
\vspace{10em}
\Huge{\textsc{\textbf{Models of Theoretical Physics}}}\\
\vspace{4em}
\LARGE{\textsc{di}}\\
\vspace{3em}
\huge{\textsc{Tommaso Tabarelli}}\\
\vspace{10em}
\LARGE{October 2018}\\
\line(1,0){450}
\end{center}

\end{titlepage}
\pagenumbering{roman}
\let\cleardoublepage\clearpage

\thispagestyle{empty}
\null
\vfill
\begin{center}
\small{This file is released under \href{http://creativecommons.org/licenses/by-sa/4.0/}{\color{blue}Creative Commons \emph{Attribution-ShareAlike 4.0 International} license}.}
\begin{figure}[H]
\centering
\href{http://creativecommons.org/licenses/by-nc-sa/4.0/deed.it}{\includegraphics[scale=0.15]{by-sa.png}}
\end{figure}
\vspace{0.5em}
This means that this material can be freely modified and redistributed, as long as it is released under the same license, and proper credit is given to the original author.\\

\vspace{2em}
\scriptsize{October 2018}
\end{center}

\newpage
\thispagestyle{empty}
\chapter*{Introduction}
The aim of this document is to collect the notes of \textit{Models of Theoretical Physics} course, held by professors Marco Baiesi and Amos Maritan for the "Physics of data" curriculum in the academic year 2018-2019 (which is the first year of this new curriculum), to have them written in a neater way.

As just told, this document is far from pretendig to be perfect and his goal is to help studying in a neater and better way. For this reason, there may be some errors among it.

I have also decided to release this document together with its \LaTeX{} source code under the \href{http://creativecommons.org/licenses/by-sa/4.0/}{\color{blue}Creative Commons \emph{Attribution-ShareAlike 4.0 International}} license. In brief this means that this document can be freely modified and redistributed, provided that it is released again under the same license and that proper credit is given to the original author.\\

I have tried to be as much ruthless as possible in finding and correcting errors and mistakes, and I apologize if some have survived.\\

I hope that the overall result will anyway be satisfactory.

\vspace{2em}

\hfill \emph{Padua, October 2018}

\hfill \emph{Tommaso Tabarelli}

\newpage
\thispagestyle{empty}
\let\cleardoublepage\clearpage

\tableofcontents
\pagenumbering{arabic}

\chapter{Methods to compute usefull integrals}

In this course we will use many kind of particular integrals. For this reason, in this first chapter we will see a brief introduction to all we need to calculate them.

\section{Gaussian integrals}

Let us consider the following integral:\\

\begin{equation}
Z(A) = \int d^{n}x \cdot e^{ -A_2(\vec{x}) }
\end{equation}

where $Z(A)$ is the integral we want to calculate, $A_2(\vec{x})$ is a x quadratic form as the following:

\begin{equation}
A_2(\vec{x}) = \frac{1}{2} \sum\limits_{i,j = 1}^{n} x_i A_{ij} x_j
\end{equation}

Here $A$ is a matrix that can be identified as a metric matrix.
In the case we consider $A$ is symmetric with (generally) complex coefficients; furthermore, it has non-negative real parts and non-vanishing eigenvalues $a_i$.

$$ Re(a_i) \geq 0 \hspace{1cm} \mathrm{and} \hspace{1cm}  a_i \neq 0_{\mathbb{C}} $$

(Note: if these conditions are not true the integral would be divergent...).\\

Let us bring $A \in \mathbb{R}$. In this case it can be digonalized with orthogonal transformation $O$, for which it holds:

$$ \sum\limits_{i}^{} O_{ij} x_j = x'_j \hspace{1cm} \mathrm{and} \hspace{1cm} |det(O)| = 1 $$

This implies that the Jacobian matrix of the transormation $J$ (which is $O$ itself) has $|det(J)| = 1$: so when we use it to change coordinates in the integral, we can say that "it has no effect" (we multiply by 1).
Furthermore, with that transformation the non-diagonal coefficients of the new matrix become all 0 (the matrix after the transformation is diagonal), so the new coordinates $x'_j$ are indipendents: this mean that we can evaluate $Z(A)$ by dividing it in the integrals of every single $x'_j$.\\
We obtain:

\begin{equation}
Z(A) = (2\pi)^{ \frac{n}{2} } \hspace{1mm} \prod\limits_{i=1}^{n} \hspace{1mm} a_i^{ -\frac{1}{2} } = (2\pi)^{ \frac{n}{2} } \hspace{1mm} (det(A))^{ -\frac{1}{2} }
\end{equation}

(Note that for a diagonal matrix the product of the eigenvalues is equal to the determinant. Moreover, for Binet's theorem it holds $det(A \cdot O) = det(A) \cdot det(O)$ (true if $A$ and $O$ are square matrix with same dimensions)).\\
(Note: I think that what just written in the previous note modifies the conditions on $det(O)$, which I think would be $det(O) = +1$ and not $|det(O)| = 1 $ otherwise the last formula is meaningless.\\
This have to be investigated.\\
However, the important key is that the integral with that tranformation does not become divergent.)


\end{document}
